%!TEX root = /Users/ego/Boulot/TKZ/tkz-fct/doc-fr/TKZdoc-fct-main.tex     
\section{Quelques exemples}      

\subsection{Variante intermédiaire : \TIKZ\ + \tkzname{tkz-fct}}
Les codes de \TIKZ\ et de \tkzname{tkz-fct} peuvent se compléter. Ainsi les axes et les textes sont gérés par \tkzname{tkz-fct} mais la courbe est laissée à \TIKZ\ et \tkzname{gnuplot}.

\bigskip

\begin{center}
	\begin{tkzexample}[]
	 \begin{tikzpicture}[scale=3]
	  \tkzInit[xmin=0,xmax=4,ymin=-1.5,ymax=1.5]
	  \tkzAxeY[label=$f(x)$]
	  \tkzDefPoint(1,0){x} \tkzDrawPoint[color=blue,size=0.6pt](x)
	  \shade[top color=gray!80,bottom color=gray!20] (1,0)%
	         plot[id=ln,domain=1:2.718] function{log(x)} |-(1,0);
	  \draw[color=blue] plot[id=ln,domain=0.2:4,samples=200]function{log(x)};
	  \tkzAxeX
	  \tkzText[draw,color= black,fill=brown!50](2,-1)%
	          {$\mathcal{A} = \int_1^{\text{e}}\ln(x)\text{d}x =%
	            \big[x\ln(x)\big]_{1}^{\text{e}} = \text{e}$}
	  \tkzText[draw,color= black,fill=brown!50](2,0.3){$\mathcal{A}$}   
	 \end{tikzpicture}
	\end{tkzexample} 
\end{center}

 \newpage  
 \subsection{Courbes de \tkzname{Lorentz}}

 $f(x)=\dfrac{\text{e}^x-1}{\text{e}-1}$ et $g(x)=x^3$ 
 

\begin{center}
\begin{tkzexample}[vbox]
\begin{tikzpicture}[scale=1]
  \tkzInit[xmax=1,ymax=1,xstep=0.1,ystep=0.1]
  \tkzGrid(0,0)(1,1)
  \tkzAxeXY
  \tkzFct[color   = red,domain = 0:1]{(exp(\x)-1)/(exp(1)-1)}
  \tkzDrawTangentLine[kl=0,kr=0.4,color=red](0)
  \tkzDrawTangentLine[kl=0.2,kr=0,color=red](1)
  \tkzText[draw,color  = red,fill = brown!30](0.4,0.6)%
          {$f(x)=\dfrac{\text{e}^x-1}{\text{e}-1}$}
  \tkzFct[color   = blue,domain = 0:1]{\x*\x*\x}
  \tkzDrawTangentLine[kl=0,kr=0.4,color=blue](0)
  \tkzDrawTangentLine[kl=0.2,kr=0,color=blue](1)
  \tkzText[draw,color = blue,fill = brown!30](0.8,0.1){$g(x)=x^3$}
  \tkzFct[color = orange,style = dashed,domain = 0:1]{\x}
  \tkzDrawAreafg[between=c and b,color=blue!40,domain = 0:1]
  \tkzDrawAreafg[between=c and a,color=red!60,domain = 0:1]
\end{tikzpicture}
\end{tkzexample}   
\end{center}


 \newpage  
 \subsection{Courbe exponentielle}  
   $f(x) = (-x^2+x+2)\exp(x)$ 
   
\begin{center}
\begin{tkzexample}[small] 
\begin{tikzpicture}[scale=1.25]   
   \tkzInit[xmin=-6,xmax=4,ymin=-5,ymax=6]
   \tkzGrid
   \tkzAxeXY  
   \tkzFct[color=red,thick,domain=-6:2.1785]{(-x*x+x+2)*exp(x)}
   \tkzSetUpPoint[size=6]
   \tkzDrawTangentLine[draw,kl=2](0)
   \tkzDefPoint(2,0){b}  \tkzDrawPoint(b)
   \tkzDefPoint(-1,0){c} \tkzDrawPoint(c)
   \tkzText(2,4){($\mathcal{C}$)}
   \tkzText(-2,-3){($\mathcal{T}$)}
\end{tikzpicture}
\end{tkzexample} 
\end{center}


 \subsection{Axe logarithmique} 
\begin{tkzexample}[vbox]
\begin{tikzpicture}[scale=0.8]
 \tkzInit[xmax=14,ymax=12]
 \draw[thin,->] (0,0) -- (14,0) node[below left] {};
 \draw[thin,->] (0,0) -- (0,12) node[below left] {};
 \foreach \x/\xtext in {0/0,2/10,4/20,6/30,8/40,10/50,12/60,14/70}%
     {\draw[shift={(\x,0)}] node[below] {$\xtext$ };}
 \foreach \y/\z in {0/0,3/1,6/2,9/3,12/4}%
     {\draw[shift={(0,\y)}] node[left] {$10^{\z}$};}
 \foreach \x in {1,2,...,14}{\tkzVLine[gray,thin]{\x}}   
 \foreach \y in {3,6,...,12}{\tkzHLine[gray,thin]{\y}}
 \foreach \y in {0,3,...,9}{
 \foreach \z in {0.903,1.431,1.806,2.097,2.334,2.535,2.709,2.863}%
   {\tkzHLine[thin,gray,shift={(0,\y)}] {\z}}}  
 \tkzDefPoint(0,6.90){a}
 \tkzDefPoint(10,9.30){b}
 \tkzDrawPoints(a,b)
 \tkzLabelPoint(a){$M_{1}$}
 \tkzLabelPoint(b){$M_{11}$}
\end{tikzpicture}
\end{tkzexample}

 \subsection{Un peu de tout} 
\begin{tkzexample}[vbox]   
\begin{tikzpicture}[scale=.8]
 \tkzInit[xmin=5,xmax=40,ymin=0,ymax=350,xstep=2.5,ystep=25]
 \tkzAxeX[label=$q$]
 \tkzAxeY[label=$C(q)$] \tkzGrid
 \tkzFct[domain=5:40]{0.1*\x**2+2*\x+60}
 \foreach \vv in {5,10,...,40}{%
   \tkzDefPointByFct(\vv)
    \tkzDrawPoint(tkzPointResult)}
 \tkzFct[domain=5:40]{(108*log(\x)-158)}
 \tkzText(37.5,280){$C$} 
 \tkzText(37.5,220){$R$}
 \tkzDefSetOfPoints{%
 5/15,10/90,15/135,20/170,25/190,30/200,35/230,40/240}
\tkzDrawSetOfPoints[mark = x,mark size=3pt]   
\end{tikzpicture}
\end{tkzexample}   
\endinput